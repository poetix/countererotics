\documentclass[ebook]{memoir}

\title{Counter-Erotics}
\author{Dominic Fox}
\date{March 2nd, 2013}

\begin{document}
\maketitle

\chapter{What did Radical Feminism Think?}

Doctrine of roots and rudiments, of first and last things, a radical theory is never content with the explication of appearances. It has to go below the surface, behind the scenes, in search of the hidden power that organises the visible. In order to be a radical, one must start with a vision of the world as split in two: on one side of the split is a world of actualities, in which things "just are" a certain way, and on the other side is a world of forces, which shapes the world of actualities to secret and usually nefarious ends. The radical is not principally interested in rearranging the world of actualities, but in struggling with the world of forces. For the radical, to live out this struggle and know what was finally at stake in it would be to live and know \emph{truly}.

The radical perceives that the world is wounded, and affirms that the wound can be named. But so deep is the wound that its name is practically a synonym for the name of the world itself, so that the attempt to heal the wound can become synonymous with the attempt to undo the entire actual configuration of the world. Thus, in a famous formulation, the Communist Manifesto names communism as "the real movement which abolishes the existing state of things". Because it is only the half-world of actualities that risks abolition, or revolutionary reconfiguration, in order that the "real movement" may realise itself, the radical demand for total change and upheaval can mask a surprisingly placid vision of the true nature of things. As the Situationists declared: beneath the paving stones, wrenched up and hurled at the police, lies the beach.

\end{document}
