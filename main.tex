\documentclass[ebook]{memoir}

\title{Counter-Erotics}
\author{Dominic Fox}
\date{March 2nd, 2013}

\begin{document}
\maketitle

\chapter{What did Radical Feminism Think?}

\section{Radical feminism as \emph{a} radicalism}

Doctrine of roots and rudiments, of first and last things, a radical theory is never content with the enumeration of appearances. It has to go below the surface, behind the scenes, in search of the hidden power that organises the visible. In order to be a radical, one must start with a vision of the world as split in two: on one side of the split is the world of actualities, in which things "just are" a certain way, and on the other side is the world of forces, which shapes the world of actualities to secret and usually nefarious ends. The radical is not principally interested in rearranging the world of actualities, but in struggling with the world of forces. For the radical, it is necessary to live out this struggle and know what is finally at stake in it, in order to live and know \emph{truly}.

The radical perceives that the world is wounded, and affirms that the wound can be named. But so deep is the wound that its name is practically a synonym for the name of the world itself, so that the attempt to heal the wound can become synonymous with the attempt to undo the entire actual configuration of the world. Thus, in a famous formulation, the Communist Manifesto names communism as "the real movement which abolishes the existing state of things". However, because it is only the half-world of actualities that risks abolition, or revolutionary reconfiguration, in order that the "real movement" may realise itself, the radical demand for total change and upheaval can mask a surprisingly placid vision of the true nature of things. As the Situationists liked to say: beneath the paving stones, wrenched up and hurled at the police, lies the beach.

Instead of thinking of radical feminism as a variety of feminism that was in some way more radical than the others, we should perhaps think of it as a variety of radicalism that was more feminist. For radical feminism begins, like every radicalism, by splitting the world in two. On the one side there is what we will call the world of men and women, the world in which there "just are" male and female persons, and the masculinity and femininity proper to each. On the other side there is the political violence of male domination, the force which is ultimately responsible for making the world of men and women what it is. On the side of actuality, gender is the law: there are men and women. On the side of force, gender is the violence which makes the law: there is male domination.

Now, what this means is that male domination is not simply an inequality in the world of men and women, an injustice which has taken place within this world and which can be redressed by making things more equal. The violence of male domination has wounded the world by making it \emph{into} a world of men and women, a world of which gender is the law. To undo this violence, to heal the wound of gender, would mean unravelling the actual configuration of this world. Thus, like communism in the formulation of Marx and Engels, women's liberation was to be seen as "the real movement which abolishes the existing state of things": as a revolutionary struggle against the ruling powers.

\section{Rudiments of Radical Feminism}

If we had to extrapolate from the thinking of radical feminism a single precept, it would be the following: \emph{there are two genders}. Two, and only two.

This is not a conclusion reached by taking an inventory of gender, by considering all the different ways people have of being more or less gendered, more or less one thing or another or something else again. The world of gender does not permit itself to be inventoried in this way: there is always some supplement held in reserve, some as yet unattempted diagonalisation. No-one will ever know, finally, how many gendered positions there are.

A book by Judith Halberstam considers the varieties of "female masculinity", the dissemination of "masculinities" beyond the locus of powerful (white, middle class) maleness. There is no limit in principle to this dissemination, but still, we have to ask: why, finally, are all these variations on the themes of "butchness", aggression, gallantry and initiative, all these survival strategies and sexy tough-guy poses, still martialled together under the banner of the "masculine"? What is the hidden law regulating dissemination, under which even the most subversively performative refiguration of gender must remain legible in terms of the masculine/feminine system?

There are two genders because gender is not primarily an identity: gender-\emph{identification} is an operation carried out within a space already opened, and already determined, by gendering as the political act which makes the world into a world of men and women. For radical feminism, gender \emph{as such{ was political violence, the ongoing violent repetition of a primal act of violence against women: their subordination to men. Gender was to be seen as the intimate projection of male violence, male domination: its habit, its code, its discourse of legitimation.

As counter to this violence, radical feminist proposed two things. Firstly, an end to gender-identification (sometimes contemplated under the label of /emph{androgyny}, a withdrawal of consent from all of the figures and tracings of gender.Not dissemination - not female masculinity, or queer masculinity, or the anarchic proliferation of masculinit\emph{ies} - but debunking and abandonment. If feminism was "the women's movement", this was a \emph{political} identification under which those wronged by violence-against-women were to unite their struggles. One would have to detach oneself from a falsifying, externally imposed and unconsciously accepted identity, in order to attach oneself to a militant identity that was self-created, consciously adopted, and oriented towards the truth.

Secondly, a revalorisation and rediscovery of the female body (Robin Morgan proposed the speculum as a revolutionary tool) as a site of oppression and resistance. The body would confront its gendering with its sex; it would show what had been done to it, and what it in its turn could do.
\end{document}
