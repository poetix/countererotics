\documentclass[ebook]{memoir}

\title{Counter-Erotics}
\author{Dominic Fox}
\date{March 2nd, 2013}

\begin{document}
\maketitle

\chapter{What did Radical Feminism Think?}

\section{Radical feminism as \emph{a} radicalism}

Doctrine of roots and rudiments, of first and last things, a radical theory is never content with the enumeration of appearances. It has to go below the surface, behind the scenes, in search of the hidden power that organises the visible. In order to be a radical, one must start with a vision of the world as split in two: on one side of the split is the world of actualities, in which things "just are" a certain way, and on the other side is the world of forces, which shapes the world of actualities to secret and usually nefarious ends. The radical is not principally interested in rearranging the world of actualities, but in struggling with the world of forces. For the radical, it is necessary to live out this struggle and know what is finally at stake in it, in order to live and know \emph{truly}.

The radical perceives that the world is wounded, and affirms that the wound can be named. But so deep is the wound that its name is practically a synonym for the name of the world itself, so that the attempt to heal the wound can become synonymous with the attempt to undo the entire actual configuration of the world. Thus, in a famous formulation, the Communist Manifesto names communism as "the real movement which abolishes the existing state of things". However, because it is only the half-world of actualities that risks abolition, or revolutionary reconfiguration, in order that the "real movement" may realise itself, the radical demand for total change and upheaval can mask a surprisingly placid vision of the true nature of things. As the Situationists liked to say: beneath the paving stones, wrenched up and hurled at the police, lies the beach.

Instead of thinking of radical feminism as a variety of feminism that was in some way more radical than the others, we should perhaps think of it as a variety of radicalism that was more feminist. For radical feminism begins, like every radicalism, by splitting the world in two. On the one side there is what we will call the world of men and women, the world in which there "just are" male and female persons, and the masculinity and femininity proper to each. On the other side there is the political violence of male domination, the force which is ultimately responsible for making the world of men and women what it is. On the side of actuality, gender is the law: there are men and women. On the side of force, gender is the violence which makes the law: there is male domination.

Now, what this means is that male domination is not simply an inequality in the world of men and women, an injustice which has taken place within this world and which can be redressed by making things more equal. The violence of male domination has wounded the world by making it into a world of men and women, a world of which gender is the law. To undo this violence, to heal the wound of gender, would mean unravelling the actual configuration of this world. Thus, like communism in the formulation of Marx and Engels, women's liberation was to be seen as "the real movement which abolishes the existing state of things": as a revolutionary struggle against the ruling powers.

\end{document}
